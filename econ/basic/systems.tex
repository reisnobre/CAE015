\section{Sistemas Econômicos}
Existem diferentes formas de se utilizar as respostas obtidas através dos problemas básicos do mercado. Tais diferentes respostas deram origem a diferentes sistemas econômicos que buscam solucionar os problemas econômicos da sua própria maneira.\par
Os sistemas económicos são uma parte da Economia que estuda os métodos e instituições pelas quais sociedades determinam a propriedade, direção e alocação dos recursos económicos e as suas respectivas trajetórias de desenvolvimento econômico.\par
Esses sistemas econômicos podem ser agrupados em dois grandes blocos:\par
\begin{enumerate}
    \item Sistema de economía de mercado (descentralizada)
    \item Sistema de economía planificada (centralizada)
\end{enumerate}
\subsection{Sistemas de Economia de Mercado}
Dentre as economias de mercado, podemos chegar a duas categorias de sistemas
\begin{enumerate}
    \item Sistemas de concorrência pura (sem interferência do governo)
    \item Sistemas de economía mista (com interferência governamental)
\end{enumerate}
\subsubsection{Sistemas de concorrência pura}
No sistema de economía pura, os problemas fundamentais da economia se resolvem sozinhos, pois tal, é por natureza perfeitamente competitivo. Todos as questões de o que e quanto, como e para quem produzir, são conduzidos por uma “mão invisível” segundo o conceito do laissez-faire, termo cunhado em 1681 por Jean-Baptiste Colbert junto a um grupo de empresários franceses.\par
Tais ajustes acontecem através de duas ações possíveis, caso exista excesso de oferta ou escassez de demanda, as empresas são forçadas a reduzir os seus preços para dar vazão os produtos até os estoques estejam em níveis satisfatórios, o inverso acontece quando existe um excesso na demanda, onde formam-se filas e disputa pelos bens, o preço tende a aumentar até que se atinja um nível de equilíbrio, a esses dois fenômenos é dado o nome de mecanismos de preço.\par
Esse sistema é a base da filosofia do liberalismo econômico, que prega a soberania do mercado, sem intervenção do estado.\par
Dentre seus maiores problemas se encontra que o mundo simplesmente não funciona dessa forma e portanto, não existe nenhum país que utilize o sistema de economia totalmente pura, normalmente sendo utilizado em seu lugar o sistema de economia mista.\par
\subsubsection{Sistemas de economía mista}
O sistema de economia mista nasceu no século XX, quando a economia se tornou mais complexa ao passo que os sindicatos, monopólios começaram a se desenvolveram. Após algumas ocorrências histórias dramáticas do ponto de vista econômico, ficou evidente que o mercado necessita de ajustes externos para garantir seu funcionamento saudável.\par
A intervenção do governo foi justificada, como uma medida para eliminar problemas na alocação e distribuição de recursos. Dentre essas intervenções encontram-se medidas que buscavam realizar a formação de preços, investimentos em áreas que o setor privado não possuía interesse ou capacidade de suprir, fornecer serviços públicos, para atender às necessidades da sociedade.\par
Diferente dos sistemas econômicos planificados, o governo no sistema de economia mista, não toma controle total do mercado, ele age apenas como agente ajustador do mesmo.\par
\subsection{Sistemas de Economia de Mercado}
A economia centralizada ou também conhecida como economia planificada, é um sistema econômico onde não há desordem na produção, toda sua produção é previa e racionalmente planejada por especialistas. Os meios de produção fica na responsabilidade do Estado enquanto toda atividade econômica é controlada por um Agência central ou Órgão Central de Planejamento.\par
Todo planejamento é feito de forma que, teoricamente, não haja escassez ou abundância de determinados produtos. Neste planos é definida toda a orientação económica nomeadamente, o papel dos preços no processo produtivo, as quantidades de bens a produzir, locais de produção, repartição de bens e etc. Esta forma de actuar sobre a economia exige ter noções das necessidades dos indivíduos  e dos fatores produtivos disponíveis, assim é possível encontrar respostas para os problemas fundamentais da economia(o quê, quanto e como produzir).\par
\clearpage