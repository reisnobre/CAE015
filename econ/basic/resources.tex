\section{Escassez e problemas fundamentais}
Um recurso é considerado escasso se a demanda para aquele recurso a zero preço iria exceder a sua quantidade disponível\citep{bvfd}\par
Considerando a nossa realidade, na qual temos recursos finitos e portanto escassos algumas decisões claras devem ser tomadas em detrimento do melhor uso dos mesmos. Dentre essas decisões podemos destacar as três questões fundamentais:  O que e quanto?, Como? e Para quem?.\par
\textbf{O que e Quanto} produzir: Para se obter tal resposta, é necessário uma análise da sociedade, pois vimos que Economia tem como função servir a um interesse em comum, a um indivíduo. Essa análise mostrará as quantidades e proporções que serão disponibilizados à sociedade. Esse estudo faz parte do conceito de demanda.\par
\textbf{Como} produzir: A produção é uma questão que depende primeiramente dos recursos a disposição. Esses recursos não dizem respeito apenas às matérias primas a serem utilizadas na produção de um dado produto, deve-se considerar também questões como tecnologia, mão de obra qualificada. Acima de tudo a produção deve ser viável de um ponto de vista econômico.\par
\textbf{Para quem} produzir: Enquanto a análise de mercado me é capaz de fornecer uma visão geral das necessidades de mercado e de demanda, para responder essa questão, deve-se levar em consideração que grupo específico o produtor deseja alcançar. Suponha como exemplo que através da pesquisa de mercado, você - o produtor - está ciente de uma demanda de mercado para lapiseiras (grafites). Cabe a você - o produtor - agora decidir quem será seu público. É necessário definir qual o preço, o fator determinante nessa fase, que você dará ao seu produto e assim delimitar para quem seu produto será útil.\par
Tendo em vista a nossa realidade de bens limitados e necessidades ilimitadas as análises acima se fazem indispensáveis para a produção e distribuição. Se todas as perguntas acima forem respondidas da forma correta, serão apenas produzidos bens com consumidores e que serão, de um ponto de vista econômico, viáveis.
\clearpage