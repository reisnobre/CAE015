\section{Conceitos básicos de Economia}
A Economia é definida como uma ciência social que tem como foco tratar de problemas como produção, distribuição de bens para satisfazer as nossas necessidades. Essa ciência busca encontrar formas de melhor lidar com as restrições físicas, como escassez de recursos e fatores de produção.\par
Essa escassez tem origem na diferença entre nossa capacidade de produzir e as nossas necessidades. Enquanto uma tem suas limitações por depender dos fatores de produção como mão de obra, capital, terra, matérias-primas,\citep{emm} a outra, nossas necessidades são ilimitadas, pois o crescimento populacional renova as necessidades básicas.\par
Tendo isso em mente se tal escassez não existisse, não haveria a necessidade do estudo da economia e nem de seus subtópicos como: inflação, crescimento econômico, déficit no balanço de pagamento e afins.\citep{emm}\par	
Portanto é possível concluir que o objeto de estudo da Economia de forma mais pura são os processos de escolha afim de lidar com a escassez.
\clearpage